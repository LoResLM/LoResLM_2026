\setlength{\parskip}{2ex}

We are pleased to present the proceedings of the second Workshop on Language Models for Low-Resource Languages (LoResLM 2026), co-located at the 19\textsuperscript{th} Conference of the European Chapter of the Association for Computational Linguistics (EACL 2026) in Rabat, Morocco.

% background and LoResLM initiation
Natural language processing (NLP) has experienced rapid progress in recent years, driven largely by advances in neural language models, including transformer-based architectures and large language models, which have achieved state-of-the-art performance across a wide range of tasks with diverse emerging capabilities. Despite these advances, since the effectiveness of language models (LMs) is primarily determined by the characteristics of their pre-trained language corpora, these models tend to be more focused on high-resource languages. As a result, many models often struggle to adequately support low-resource languages, which are estimated to be around 7,000. Despite their worldwide usage, these languages generally receive little research attention and lack sufficient digital data and resources to support NLP tasks. Recognising the growing need to address this imbalance and promote more inclusive language technologies, the research community has increasingly focused on developing and adapting LMs for low-resource languages. In response to this momentum, we initiated LoResLM to provide a dedicated forum for researchers to share insights, resources, and emerging work in this area. Following a successful inaugural edition, LoResLM returned this year for its second iteration, continuing to foster collaboration and advance research toward more equitable and linguistically diverse NLP.

% submission stats
Primarily focusing on developing and evaluating neural language models for low-resource languages, LoResLM 2026 invited submissions on a broad range of topics, including creating corpora, developing benchmarks, building or adapting LMs, and exploring LM applications for low-resource languages. In total, we received 79 submissions. Among these, we accepted 55 papers, including 44 long papers and 11 short papers, to appear in the workshop proceedings following the peer-review process.

% proceedings summary
The accepted papers reflect substantial linguistic and research diversity, covering low-resource languages from 13 distinct language families. The majority representation is from the Indo-European family, with contributions across nine of its branches. Overall, the studies address 82 distinct languages, highlighting the breadth of linguistic contexts explored within the workshop. In addition to this linguistic diversity, the papers span 11 research areas, with NLP and LLM Applications, Language Modelling, and Machine Translation and Translation Aids emerging as the most represented topics. We are pleased to present such a broad range of contributions, reflecting the growing momentum of research and offering promising directions for future work on low-resource languages.

% thanks
LoResLM 2026 would not be successful without several wonderful people who joined this initiative. First of all, we would like to thank the authors who submitted their work to the workshop, encouraging research in many low-resource languages that span diverse research areas. We are very grateful for the programme committee members who played a crucial role towards this workshop's success with their timely engagement with the review process, providing constructive feedback to help authors improve the quality of their papers to meet the general standards. We are also particularly thankful to Prof Barbara Plank for accepting our invitation to serve as the keynote speaker, sharing her knowledge and experience, and providing valuable insights to the NLP community. Our sincere appreciation also goes to the Artificial Intelligence Journal (AIJ) for sponsoring the workshop. We are very grateful to everybody for supporting us to make LoResLM 2026 successful.


\vspace{1em}
Hansi Hettiarachchi, Tharindu Ranasinghe, Alistair Plum, Paul Rayson, Ruslan Mitkov, Mohamed Gaber, Damith Premasiri, Fiona Anting Tan, and Lasitha Uyangodage\\
(LoResLM 2026 Organisers)

\url{https://loreslm.github.io/home}